\documentclass{article}
\usepackage{t1enc}
\usepackage[magyar]{babel}
\usepackage{amsthm}
\usepackage{algorithm}
\usepackage{listings}
\usepackage{hulipsum}
\usepackage{float}
\usepackage{geometry}
\usepackage{algpseudocode}

\title{\Huge{\LaTeX}}
\author{Buha Milán}

\theoremstyle{plain}
\newtheorem{tet}{Tétel}
\theoremstyle{definition}
\newtheorem{defin}[tet]{Definíció}
\theoremstyle{plain}
\newtheorem{lemma}[tet]{Lemma}
\theoremstyle{remark}
\newtheorem{feladat}[tet]{Feladat}
\theoremstyle{plain}
\newtheorem{bizonyitas}[tet]{Bizonyitas}

\begin{document}
	\maketitle
	\newpage
	
	\begin{tet}
	tétel
	\end{tet}
	
	\begin{tet}[Pit]
	ez is tétel
	\end{tet}
	
	\begin{defin}
	Definíció
	\end{defin}
	
	\section{Feladatok} \begin{feladat} Ez egy feladat \end{feladat}
	\section{Lemma section} \begin{lemma} Lemma \end{lemma}
	
	\begin{bizonyitas}
	Ez a bizonyítás
	\end{bizonyitas}
	
	\newpage
	\verb|\LaTeX| \verb|\LaTeX| \verb|\LaTeX|
	
	\begin{verbatim}
	\LaTeX
	Ez a bizonyítás
	\LaTeX
	\textbf{asd}
	\end{verbatim}
	
	\begin{lstlisting}[frame = single][language=python]
def binary_search(arr, val, start, end):
if start == end:
	if arr[start] > val:
		return start
	else:
		return start+1
elif start > end:
	return start
else: 
	mid = (start+end)/2
	if arr[mid] < val:
		return binary_search(arr, val, mid+1, end)
	elif arr[mid] > val:
		return binary_search(arr, val, start, mid-1)
	else: # arr[mid] = val
		return mid
			
def insertion_sort(arr):
   for i in xrange(1, len(arr)):
		val = arr[i]
		j = binary_search(arr, val, 0, i-1)
		arr = arr[:j] + [val] + arr[j:i] + arr[i+1:]
		return arr
	\end{lstlisting}
	
	\begin{algorithmic}
	\State $i \gets 10$
	\If{$i\geq 5$} 
    \State $i \gets i-1$
	\Else
  	  \If{$i\leq 3$}
        	\State $i \gets i+2$
   		\EndIf
	\EndIf 
	\end{algorithmic}

\end{document}